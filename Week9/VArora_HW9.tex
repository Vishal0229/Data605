\documentclass[]{article}
\usepackage{lmodern}
\usepackage{amssymb,amsmath}
\usepackage{ifxetex,ifluatex}
\usepackage{fixltx2e} % provides \textsubscript
\ifnum 0\ifxetex 1\fi\ifluatex 1\fi=0 % if pdftex
  \usepackage[T1]{fontenc}
  \usepackage[utf8]{inputenc}
\else % if luatex or xelatex
  \ifxetex
    \usepackage{mathspec}
  \else
    \usepackage{fontspec}
  \fi
  \defaultfontfeatures{Ligatures=TeX,Scale=MatchLowercase}
\fi
% use upquote if available, for straight quotes in verbatim environments
\IfFileExists{upquote.sty}{\usepackage{upquote}}{}
% use microtype if available
\IfFileExists{microtype.sty}{%
\usepackage{microtype}
\UseMicrotypeSet[protrusion]{basicmath} % disable protrusion for tt fonts
}{}
\usepackage[margin=1in]{geometry}
\usepackage{hyperref}
\hypersetup{unicode=true,
            pdftitle={HW9},
            pdfauthor={Vishal Arora},
            pdfborder={0 0 0},
            breaklinks=true}
\urlstyle{same}  % don't use monospace font for urls
\usepackage{color}
\usepackage{fancyvrb}
\newcommand{\VerbBar}{|}
\newcommand{\VERB}{\Verb[commandchars=\\\{\}]}
\DefineVerbatimEnvironment{Highlighting}{Verbatim}{commandchars=\\\{\}}
% Add ',fontsize=\small' for more characters per line
\usepackage{framed}
\definecolor{shadecolor}{RGB}{248,248,248}
\newenvironment{Shaded}{\begin{snugshade}}{\end{snugshade}}
\newcommand{\KeywordTok}[1]{\textcolor[rgb]{0.13,0.29,0.53}{\textbf{#1}}}
\newcommand{\DataTypeTok}[1]{\textcolor[rgb]{0.13,0.29,0.53}{#1}}
\newcommand{\DecValTok}[1]{\textcolor[rgb]{0.00,0.00,0.81}{#1}}
\newcommand{\BaseNTok}[1]{\textcolor[rgb]{0.00,0.00,0.81}{#1}}
\newcommand{\FloatTok}[1]{\textcolor[rgb]{0.00,0.00,0.81}{#1}}
\newcommand{\ConstantTok}[1]{\textcolor[rgb]{0.00,0.00,0.00}{#1}}
\newcommand{\CharTok}[1]{\textcolor[rgb]{0.31,0.60,0.02}{#1}}
\newcommand{\SpecialCharTok}[1]{\textcolor[rgb]{0.00,0.00,0.00}{#1}}
\newcommand{\StringTok}[1]{\textcolor[rgb]{0.31,0.60,0.02}{#1}}
\newcommand{\VerbatimStringTok}[1]{\textcolor[rgb]{0.31,0.60,0.02}{#1}}
\newcommand{\SpecialStringTok}[1]{\textcolor[rgb]{0.31,0.60,0.02}{#1}}
\newcommand{\ImportTok}[1]{#1}
\newcommand{\CommentTok}[1]{\textcolor[rgb]{0.56,0.35,0.01}{\textit{#1}}}
\newcommand{\DocumentationTok}[1]{\textcolor[rgb]{0.56,0.35,0.01}{\textbf{\textit{#1}}}}
\newcommand{\AnnotationTok}[1]{\textcolor[rgb]{0.56,0.35,0.01}{\textbf{\textit{#1}}}}
\newcommand{\CommentVarTok}[1]{\textcolor[rgb]{0.56,0.35,0.01}{\textbf{\textit{#1}}}}
\newcommand{\OtherTok}[1]{\textcolor[rgb]{0.56,0.35,0.01}{#1}}
\newcommand{\FunctionTok}[1]{\textcolor[rgb]{0.00,0.00,0.00}{#1}}
\newcommand{\VariableTok}[1]{\textcolor[rgb]{0.00,0.00,0.00}{#1}}
\newcommand{\ControlFlowTok}[1]{\textcolor[rgb]{0.13,0.29,0.53}{\textbf{#1}}}
\newcommand{\OperatorTok}[1]{\textcolor[rgb]{0.81,0.36,0.00}{\textbf{#1}}}
\newcommand{\BuiltInTok}[1]{#1}
\newcommand{\ExtensionTok}[1]{#1}
\newcommand{\PreprocessorTok}[1]{\textcolor[rgb]{0.56,0.35,0.01}{\textit{#1}}}
\newcommand{\AttributeTok}[1]{\textcolor[rgb]{0.77,0.63,0.00}{#1}}
\newcommand{\RegionMarkerTok}[1]{#1}
\newcommand{\InformationTok}[1]{\textcolor[rgb]{0.56,0.35,0.01}{\textbf{\textit{#1}}}}
\newcommand{\WarningTok}[1]{\textcolor[rgb]{0.56,0.35,0.01}{\textbf{\textit{#1}}}}
\newcommand{\AlertTok}[1]{\textcolor[rgb]{0.94,0.16,0.16}{#1}}
\newcommand{\ErrorTok}[1]{\textcolor[rgb]{0.64,0.00,0.00}{\textbf{#1}}}
\newcommand{\NormalTok}[1]{#1}
\usepackage{graphicx,grffile}
\makeatletter
\def\maxwidth{\ifdim\Gin@nat@width>\linewidth\linewidth\else\Gin@nat@width\fi}
\def\maxheight{\ifdim\Gin@nat@height>\textheight\textheight\else\Gin@nat@height\fi}
\makeatother
% Scale images if necessary, so that they will not overflow the page
% margins by default, and it is still possible to overwrite the defaults
% using explicit options in \includegraphics[width, height, ...]{}
\setkeys{Gin}{width=\maxwidth,height=\maxheight,keepaspectratio}
\IfFileExists{parskip.sty}{%
\usepackage{parskip}
}{% else
\setlength{\parindent}{0pt}
\setlength{\parskip}{6pt plus 2pt minus 1pt}
}
\setlength{\emergencystretch}{3em}  % prevent overfull lines
\providecommand{\tightlist}{%
  \setlength{\itemsep}{0pt}\setlength{\parskip}{0pt}}
\setcounter{secnumdepth}{0}
% Redefines (sub)paragraphs to behave more like sections
\ifx\paragraph\undefined\else
\let\oldparagraph\paragraph
\renewcommand{\paragraph}[1]{\oldparagraph{#1}\mbox{}}
\fi
\ifx\subparagraph\undefined\else
\let\oldsubparagraph\subparagraph
\renewcommand{\subparagraph}[1]{\oldsubparagraph{#1}\mbox{}}
\fi

%%% Use protect on footnotes to avoid problems with footnotes in titles
\let\rmarkdownfootnote\footnote%
\def\footnote{\protect\rmarkdownfootnote}

%%% Change title format to be more compact
\usepackage{titling}

% Create subtitle command for use in maketitle
\newcommand{\subtitle}[1]{
  \posttitle{
    \begin{center}\large#1\end{center}
    }
}

\setlength{\droptitle}{-2em}

  \title{HW9}
    \pretitle{\vspace{\droptitle}\centering\huge}
  \posttitle{\par}
    \author{Vishal Arora}
    \preauthor{\centering\large\emph}
  \postauthor{\par}
      \predate{\centering\large\emph}
  \postdate{\par}
    \date{October 26, 2019}


\begin{document}
\maketitle

\subsection{Q 1. The price of one share of stock in the Pilsdorff Beer
Company (see Exercise 8.2.12) is given by Yn on the nth day of the year.
Finn observes that the differences Xn = Yn+1 ??? Yn appear to be
independent random variables with a common distribution having mean = 0
and variance VAR = 1/4. If Y1 = 100, estimate the probability that Y365
is}\label{q-1.-the-price-of-one-share-of-stock-in-the-pilsdorff-beer-company-see-exercise-8.2.12-is-given-by-yn-on-the-nth-day-of-the-year.-finn-observes-that-the-differences-xn-yn1-yn-appear-to-be-independent-random-variables-with-a-common-distribution-having-mean-0-and-variance-var-14.-if-y1-100-estimate-the-probability-that-y365-is}

\begin{enumerate}
\def\labelenumi{(\alph{enumi})}
\item
  \begin{quote}
  = 100.
  \end{quote}
\end{enumerate}

\begin{Shaded}
\begin{Highlighting}[]
\KeywordTok{pnorm}\NormalTok{(}\DecValTok{100} \OperatorTok{-}\StringTok{ }\DecValTok{100}\NormalTok{, }\DataTypeTok{mean =} \DecValTok{0}\NormalTok{, }\DataTypeTok{sd =} \KeywordTok{sqrt}\NormalTok{(}\FloatTok{91.25}\NormalTok{), }\DataTypeTok{lower.tail =} \OtherTok{FALSE}\NormalTok{)}
\end{Highlighting}
\end{Shaded}

\begin{verbatim}
## [1] 0.5
\end{verbatim}

\begin{enumerate}
\def\labelenumi{(\alph{enumi})}
\setcounter{enumi}{1}
\item
  \begin{quote}
  = 110.
  \end{quote}
\end{enumerate}

\begin{Shaded}
\begin{Highlighting}[]
\KeywordTok{pnorm}\NormalTok{(}\DecValTok{110} \OperatorTok{-}\StringTok{ }\DecValTok{100}\NormalTok{, }\DataTypeTok{mean =} \DecValTok{0}\NormalTok{, }\DataTypeTok{sd =} \KeywordTok{sqrt}\NormalTok{(}\FloatTok{91.25}\NormalTok{), }\DataTypeTok{lower.tail =} \OtherTok{FALSE}\NormalTok{)}
\end{Highlighting}
\end{Shaded}

\begin{verbatim}
## [1] 0.1475849
\end{verbatim}

\begin{enumerate}
\def\labelenumi{(\alph{enumi})}
\setcounter{enumi}{2}
\item
  \begin{quote}
  = 120.
  \end{quote}
\end{enumerate}

\begin{Shaded}
\begin{Highlighting}[]
\KeywordTok{pnorm}\NormalTok{(}\DecValTok{120} \OperatorTok{-}\StringTok{ }\DecValTok{100}\NormalTok{, }\DataTypeTok{mean =} \DecValTok{0}\NormalTok{, }\DataTypeTok{sd =} \KeywordTok{sqrt}\NormalTok{(}\FloatTok{91.25}\NormalTok{), }\DataTypeTok{lower.tail =} \OtherTok{FALSE}\NormalTok{)}
\end{Highlighting}
\end{Shaded}

\begin{verbatim}
## [1] 0.01814355
\end{verbatim}

\subsection{Q 2. Calculate the expected value and variance of the
binomial distribution using the moment generating
function.}\label{q-2.-calculate-the-expected-value-and-variance-of-the-binomial-distribution-using-the-moment-generating-function.}

\[g(t)=\sum _{ j=0 }^{ n }{ { e }^{ tj } } (n!/(n-j)!j!){ p }^{ j }{ q }^{ n-j }\]
\[g(t)=\sum _{ j=0 }^{ n }{ (n!/(n-j)!j!) } { { (pe }^{ t }) }^{ j }{ q }^{ n-j }\]
\[g(t)={ ({ pe }^{ t }+q) }^{ n }\]
\[g'(t)=n{ ({ pe }^{ t }+q) }^{ n-1 }{ pe }^{ t }\]

\[g''(t)=n(n-1)({ pe }^{ t }+q){ ({ pe }^{ t }) }^{ 2 }+n{ ({ pe }^{ t }+q) }^{ n }{ pe }^{ t }  \]
\[g'(0)={ n(p+q) }^{ n-1 }p=np\]
\[g''(0)=\quad { n(n-1)(p+q) }^{ n-1 }{ p }^{ 2 }+{ n(p+q) }^{ n }p\]
\[g''(0)=n(n-1){ p }^{ 2 }+np\] \[\mu ={ \mu  }_{ 1 }=g'(0)=np\]
\[{ \sigma  }^{ 2 }={ \mu  }_{ 2 }-{ \mu  }_{ 1 }^{ 2 }1=g''(0)-{ g'(0) }^{ 2 }\\ { \sigma  }^{ 2 }=n(n-1){ p }^{ 2 }+np-{ (np) }^{ 2 }\\ { \sigma  }^{ 2 }=np[(n-1)p+1-np]\\ { \sigma  }^{ 2 }=np[(np-p)+1-np]\]
\[{ \sigma  }^{ 2 }=np[1-p]\] \#\# Q 3.Calculate the expected value and
variance of the exponential distribution using the moment generating
function.

The exponential distribution probability density function is:

\[\lambda { e }^{ -\lambda x }\] Moment generating function for the
binomial distribution is:

\[g(t)=\frac { \lambda }{ \lambda-t } \quad for\quad t<\lambda\]

First Derivative,

\[g'(t)=\frac {\lambda }{ { (\lambda-t) }^{ 2 } } \] Determining mean
??1, by evaluating for t=0,

\[g'(0)=\frac { 1 }{\lambda} \]

Second Derivative,

\[g''(t)=\frac { 2{\lambda }{ { ({\lambda-t) }^{ 3 } } \] Determining
??2, by evaluating for t=0,

\[g''(0)=\frac { 2 }{ { \lambda  }^{ 2 } } } \]

variance is

\[{ \mu  }^{ 2 }-{ \mu  }_{ 1 }^{ 2 }=\frac { 2 }{ { \lambda  }^{ 2 } } -{ (\frac { 1 }{ \lambda  } ) }^{ 2 }=\frac { 2 }{ \lambda ^{ 2 } } -\frac { 1 }{ \lambda ^{ 2 } } =\frac { 1 }{ \lambda ^{ 2 } }  \]


\end{document}
